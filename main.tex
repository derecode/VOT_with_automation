\documentclass[12pt,a4paper,british]{article}
\usepackage{mathptmx}
\renewcommand{\familydefault}{\rmdefault}
\usepackage[T1]{fontenc}
\usepackage[latin9]{inputenc}
\usepackage{xcolor}
\usepackage{verbatim}
\usepackage{enumitem}
\usepackage{amsmath}
\usepackage{amsthm}
\usepackage{graphicx}
\usepackage{setspace}
\onehalfspacing

\makeatletter

%%%%%%%%%%%%%%%%%%%%%%%%%%%%%% LyX specific LaTeX commands.
\pdfpageheight\paperheight
\pdfpagewidth\paperwidth


%%%%%%%%%%%%%%%%%%%%%%%%%%%%%% Textclass specific LaTeX commands.
\theoremstyle{definition}
\newtheorem{defn}{\protect\definitionname}
\theoremstyle{plain}
\newtheorem{thm}{\protect\theoremname}
\ifx\proof\undefined\
  \newenvironment{proof}[1][\proofname]{\par
    \normalfont\topsep6\p@\@plus6\p@\relax
    \trivlist
    \itemindent\parindent
    \item[\hskip\labelsep
          \scshape
      #1]\ignorespaces
  }{%
    \endtrivlist\@endpefalse
  }
  \providecommand{\proofname}{Proof}
\fi
\theoremstyle{plain}
\newtheorem{prop}{\protect\propositionname}
\newlist{casenv}{enumerate}{4}
\setlist[casenv]{leftmargin=*,align=left,widest={iiii}}
\setlist[casenv,1]{label={{\itshape\ \casename} \arabic*.},ref=\arabic*}
\setlist[casenv,2]{label={{\itshape\ \casename} \roman*.},ref=\roman*}
\setlist[casenv,3]{label={{\itshape\ \casename\ \alph*.}},ref=\alph*}
\setlist[casenv,4]{label={{\itshape\ \casename} \arabic*.},ref=\arabic*}

%%%%%%%%%%%%%%%%%%%%%%%%%%%%%% User specified LaTeX commands.
\usepackage{lmodern}
\usepackage[a4paper, margin=1.1in]{geometry}
%\geometry{verbose,tmargin=3cm,bmargin=3cm,lmargin=2.5cm,rmargin=2.5cm}
\usepackage{xcolor}
\usepackage{prettyref}
\usepackage{graphicx}
\usepackage{setspace}
\usepackage[authoryear]{natbib}
\setstretch{1.5}
\usepackage[unicode=true,
            bookmarks=true,
            bookmarksnumbered=true,
            bookmarksopen=false, 
            breaklinks=false,
            pdfborder={0 0 1},
            backref=false,
            colorlinks=true]{hyperref}
\hypersetup{final,
            bookmarksopen,
            bookmarksnumbered,
            urlcolor={blue},
            linkcolor={blue},
            citecolor={blue},
            pdfstartview={XYZ null null fitH}}

\makeatletter
%%%%%%%%%%%%%%%%%%%%%%%%%%%%%% User specified LaTeX commands.
\pdfminorversion=4
\usepackage{etex}
\usepackage{color}
\usepackage{setspace}
\usepackage{booktabs}
%\usepackage[a4paper, margin=2in]{geometry}
\usepackage{amsfonts}
%\usepackage{amsmath}
\usepackage{bm}
\usepackage{mathrsfs}
\usepackage{multirow}
\usepackage{multicol}
\usepackage{rccol}
%\usepackage{enumitem}
%\usepackage{enumerate}
\usepackage[protrusion=true,expansion]{microtype}
\usepackage{eurosym}
\usepackage{graphicx}
%\bibliography{avrefs}
%\setlength{\abovedisplayskip}{3pt}
%\setlength{\belowdisplayskip}{3pt}
\PassOptionsToPackage{nodisplayskipstretch}{setspace}

\makeatother

\usepackage{babel}
\providecommand{\casename}{Case}
\providecommand{\definitionname}{Definition}
\providecommand{\propositionname}{Proposition}
\providecommand{\theoremname}{Theorem}

\begin{document}
\title{Allocation of time and the values of travel time and reliability with
automated cars}
\maketitle
\begin{abstract}
Self-driving cars make it increasingly possible to carry out some
activities while travelling. This paper analyses a commuter's optimal
allocation of total time among different activities and the implication
of increased productivity of in-vehicle time on the values of travel
time and reliability.
\end{abstract}

\section{Introduction}

Mobile communication devices are transforming the experience of travel,
making it possible to work, play games or watch videos while travelling
on planes, trains and buses. The car industry promises to free car
drivers from driving, thereby enabling an even more drastic transformation
of car travel. This paper explores the observation that the new technologies
have in common that they make it possible to carry out activities
while travelling that substitute for activities elsewhere, at home
or at work.

In-vehicle productivity increases when it becomes possible to do things
while travelling that were not possible before. This has happened
for passengers in trains and buses, who now can use mobile devices
to do new things. It is happening also in planes where internet access
during flight is gradually becoming available. Of course, car manufacturers
are promising that soon car drivers can be relieved from driving.

This paper looks at the impact of increasing in-vehicle productivity
on allocation of commuter's time across different activities and the
values of time and reliability. This is of fundamental importance
in transport economics and modelling. The values of travel time and
reliability are important behavioural quantities as they represent
the lion's share of benefits from large infrastructure projects. 

We built a model that examines how a commuter optimally allocates
his/her time budget among different a activities where some of these
activities can only be carried out at home or work. We showed existence
of an optimal allocation of time and analysed the optimal scheduling
choice in the presence of deterministic and random travel times. We
also examined how the values of travel time and reliability change
with increasing productivity of in-vehicle time.

The rest of the paper is organised as follows. The next section sets
out the model set up and foundation for the rest of the paper. The
third and fourth sections analyse the optimal allocation of time among
different activities and the value of travel time, value of reliability
and value of headway when trip duration is random or deterministic.
The final section provides a summary of our findings and concluding
remarks.

\section{The model}

Consider a commuter who begins a day at home and who has to take a
trip to a workplace, where he finishes the day. The commuter intends
to allocate the total available time among different activities. We
use the term home-based activity to refer to those activities that
are performed only at home and , Travelling is one of these activities
and hence part of the commuter's total time will be set aside for
this purpose. Activities

The commuter has to put aside a part of the total available time for
travelling while the remaining time can be allocated among different
activities. We categorise these activities into three Let $t_{h}$
and $t_{w}$ respectively denote the time allocated to the home-based
and work-based activities such that $0<t_{h}\leq t_{d}$ and $0<t_{w}\leq Q-t_{d}-T$.
Hence, $T+\left(t_{d}-t_{h}\right)+\left(Q-t_{d}-T-t_{w}\right)$
units of time will be available to carry out the mobile activity.
The mobile activity is fully productive at home or at work but may
be less so while travelling. This is controlled by a factor $\alpha\in\left[0,1\right]$,
such that the effective amount of the mobile activity is $t_{m}=Q-\left(1-\alpha\right)T-t_{h}-t_{w}$.
The parameter $\alpha$ indicates the productivity of in-vehicle time
relative to time at work or at home.

Consider a commuter who begins a day at home and who has to take a
trip to a workplace where he or she ends the day. The commuter has
a time budget of $Q$ time units in the clock time interval $\left[0,Q\right]$,
which is divided into $0<t_{d}<Q$ units of time spent at home; $0<T<Q$
units of time spent commuting to work; and $Q-t_{d}-T$ units of time
spent at work, where $t_{d}$ the is departure time for the commute
trip. Let $t_{h}$ and $t_{w}$ respectively denote the time allocated
to the home-based and work-based activities such that $0<t_{h}\leq t_{d}$
and $0<t_{w}\leq Q-t_{d}-T$. Hence, $T+\left(t_{d}-t_{h}\right)+\left(Q-t_{d}-T-t_{w}\right)$
units of time will be available to carry out the mobile activity.
The mobile activity is fully productive at home or at work but may
be less so while travelling. This is controlled by a factor $\alpha\in\left[0,1\right]$,
such that the effective amount of the mobile activity is $t_{m}=Q-\left(1-\alpha\right)T-t_{h}-t_{w}$.
The parameter $\alpha$ indicates the productivity of in-vehicle time
relative to time at work or at home.

Travelling is an additional activity but it does not lead to the production
of any output in itself. However, to the extent in-vehicle time is
productive, travel time can be used to perform the mobile activity.
In this sense, commuting and the mobile activity are complementary
in that they can be produced at the same time.

\textbackslash\textbackslash{}

The commuter allocates the total time among three activities

and he or she has preferences over outcomes that are produced from
the three activities.

The commuter allocates the total The commuter has preferences over
outcomes of activities The commuter derives utility from a home-based
activity performed at the origin, from a work-based activity carried
out at the destination, and from a mobile activity that may be carried
out at either location and while travelling.\footnote{The mobile activity can be thought of as being an entertainment activity
or a part of the home-based and/or work-based activities that can
be performed while driving. The home-based activity comprises of tasks
that can be performed only at home while work-based activity includes
tasks performed at the destination of the trip.}

The commuter has preferences over outcomes from these activities.

//

The commuter has a money-metric utility: %
\begin{comment}
Reformulate in the context of \textsc{Small and Verhoef} {[}2007{]}
\{\textsc{2.6.1 Value of time: basic theory}\}
\end{comment}
{} 
\begin{equation}
U\left(t_{h},t_{w};T\right)=U_{h}\left(t_{h}\right)+U_{w}\left(t_{w}\right)+U_{m}\left(Q-\left(1-\alpha\right)T-t_{h}-t_{w}\right),\label{utility}
\end{equation}
which is separable into three components depending on the activity
in which time is spent. It is assumed that each utility component
$U_{i}$ for $i=\left\{ h,w,m\right\} $ is an increasing and strictly
concave function on the range of feasible values for $t_{i}$ and
that each $U_{i}$ is twice-continuously differentiable. We assume
that $U_{h}^{\prime}\left(0\right)=U_{w}^{\prime}\left(0\right)=\infty$
and $U_{m}^{\prime}\left(0\right)<\infty$, such that the commuter
may choose not to carry out any of the mobile activity.

The commuter maximises utility choosing $\left(t_{h},t_{w}\right)$
subject to the constraint that time spent on all activities, including
travelling, be within total time available, $Q$: 
\begin{equation}
t_{h}+t_{w}+T\leq Q,\label{constraint}
\end{equation}
The constraint also indicates that the some of the time at home and/or
at work may be allocated to the mobile activity. This depends on the
status of the time constraint. Accordingly, if the time constraint
is binding, then the mobile activity is carried out only while travelling.
If however the constraint is non-binding, then some of the time at
home and/or at work will be used to perform the mobile activity. 

The commuter aims to optimally allocate his/her total time among different
activities. Given the optimal allocation of time, the commuter will
also choose the optimal departure time for the commute trip.

CUTS

Three activities (h, m, w); two types of activities (immobile and
mobile); preferences: a unit time in the home-based and work-based
activities produces one unit of output, while a unit of time in the
mobile activity produces one or $\alpha$ unit of m depending on where
it is spent. Preference over outcomes, meaning

Each activity leads to the production of a different output -- each
unit 

\section{Allocation of time and the value of travel time with deterministic
trip duration}

\subsection{Optimal allocation of time}
\begin{defn}
(\textbf{time allocation}) A time allocation is a pair $\left(t_{h},t_{w}\right)$
where each $0\leq t_{i}\leq Q$ indicating a commuter's choice of
time allocated to the home-based and work-based activities. A time
allocation is said to be \textbf{\textit{feasible}} if the time allocated
to the home-based and work-based activities does not exceed the commuter's
time budget less travel time.
\end{defn}
The commuter's problem is to optimally allocate his/her total time
among different activities given the fixed travel time. The problem
amounts to maximising trip utility subject to the constraint that
time spent on different activities cannot exceed the total available
time:
\begin{gather*}
\max_{t_{h},t_{w}}U\left(t_{h},t_{w};T\right)=U_{h}\left(t_{h}\right)+U_{w}\left(t_{w}\right)+U_{m}\left(Q-\left(1-\alpha\right)T-t_{h}-t_{w}\right)\\
\mbox{s.t. }T+t_{h}+t_{w}\leq Q
\end{gather*}
 The Lagrangian for the utility maximisation problem can be written
as
\[
\mathcal{L}\equiv U\left(t_{h},t_{w};T\right)+\lambda\left(Q-T-t_{h}-t_{w}\right)
\]
where $\lambda\geq0$ is a Lagrangian multiplier that indicates the
sensitivity of optimal utility with respect to small changes in the
time constraint. The first-order conditions for the utility maximisation
problem are as follows:

\begin{equation}
\begin{aligned}U_{h}^{\prime}\left(t_{h}\right)-U_{m}^{\prime}\left(Q-\left(1-\alpha\right)T-t_{h}-t_{w}\right)-\lambda & =0\\
U_{w}^{\prime}\left(t_{w}\right)-U_{m}^{\prime}\left(Q-\left(1-\alpha\right)T-t_{h}-t_{w}\right)-\lambda & =0\\
\lambda\left(Q-T-t_{h}-t_{w}\right) & =0\\
\lambda,t_{h},t_{w} & \geq0
\end{aligned}
\label{eq:foc_deterministic}
\end{equation}
These conditions imply the following properties hold at optimum. Firstly,
optimal time allocation with binding time constraint requires the
marginal utility of time in the mobile activity be lower than that
in each of the other two activities. The resulting difference in utility
will be equal to the resource value of a unit of time, $\lambda$.
If the time constraint is non-binding, i.e. $\lambda=0$, optimal
time allocation requires that the marginal utility of time be equalised
among the three activities. In addition, regardless of whether the
time constraint is binding, the commuter allocates the total time
in such a way that the marginal utility of time is equalised between
the home-based and the work-based activities:

\begin{equation}
U_{h}^{\prime}\left(t_{h}^{\ast}\right)-U_{w}^{\prime}\left(t_{w}^{\ast}\right)=0\label{eq:Uh_eq_Uw}
\end{equation}
where $\left(t_{h}^{\ast},t_{w}^{\ast}\right)$ denotes the optimal
time allocated to the home-based and work-based activities with the
corresponding time allocated to the mobile activity being $t_{m}^{\ast}=t_{m}\left(t_{h}^{\ast},t_{w}^{\ast}\right)$.
The following theorem establishes the existence of such an optimum.
\begin{thm}
If each $U_{i}\left(\cdot\right)$ is increasing, twice continuously
differentiable and strictly concave, then there exists a unique optimal
allocation $\left(t_{h}^{\ast},t_{w}^{\ast}\right)$ and a multiplier
\textup{$\lambda^{\ast}\geq0$} satisfying the conditions in (\ref{eq:foc_deterministic}).
\end{thm}
\begin{proof}
Since $U$ is continuous, differentiable and strictly concave on a
closed interval, it attains a unique maximum at a point $\left(t_{h}^{\ast},t_{w}^{\ast}\right)=\left\{ \left(t_{h},t_{w}\right):\frac{\partial U\left(t_{h}^{\ast},t_{w}^{\ast}\right)}{\partial t_{h}}=\frac{\partial U\left(t_{h}^{\ast},t_{w}^{\ast}\right)}{\partial t_{w}}=0\right\} $.
Since $g\left(t_{h},t_{w}\right)=-\left(Q-T-t_{h}-t_{w}\right)$ is
linear, the feasible set $\Omega=\left\{ \left(t_{h},t_{w}\right);g\left(t_{h},t_{w}\right)\leq0\right\} $
is convex. To prove the existence, suppose the constraint is inactive
at optimum and hence $\left(t_{h}^{\ast},t_{w}^{\ast}\right)$ is
in the interior of $\Omega$. Thus, it must be that $\frac{\partial U\left(t_{h}^{\ast},t_{w}^{\ast}\right)}{\partial t_{h}}=\frac{\partial U\left(t_{h}^{\ast},t_{w}^{\ast}\right)}{\partial t_{w}}=0$
and conditions in (\ref{eq:foc_deterministic}) hold with $\lambda=\lambda^{\ast}=0$.
If the constraint is active at optimum, then $g\left(t_{h}^{\ast},t_{w}^{\ast}\right)=0$.
By Farkas' Lemma, there must exist $\lambda\geq0$ such that (\ref{eq:foc_deterministic})
holds. For $\lambda^{\ast}\geq0$, the Lagrangian evaluated at any
$\left(t_{h},t_{w}\right)\in\Omega$: 
\[
\mathcal{L}\left(t_{h},t_{w},\lambda^{\ast}\right)=U\left(t_{h},t_{w}\right)-\lambda^{\ast}g\left(t_{h},t_{w}\right)
\]
is also strictly concave in $\left(t_{h},t_{w}\right)$. By observing
that $\lambda^{\ast}g\left(t_{h},t_{w}\right)\leq0$, we have
\begin{align*}
U\left(t_{h},t_{w};T\right)< & \mathcal{L}\left(t_{h},t_{w},\lambda^{\ast}\right)\\
< & \mathcal{L}\left(t_{h}^{\ast},t_{w}^{\ast},\lambda^{\ast}\right)+\frac{\partial\mathcal{L}\left(t_{h}^{\ast},t_{w}^{\ast},\lambda^{\ast}\right)}{\partial t_{h}}\left(t_{h}-t_{h}^{\ast}\right)+\frac{\partial\mathcal{L}\left(t_{h}^{\ast},t_{w}^{\ast},\lambda^{\ast}\right)}{\partial t_{w}}\left(t_{w}-t_{w}^{\ast}\right)\\
= & U\left(t_{h}^{\ast},t_{w}^{\ast};T\right)+\frac{\partial\mathcal{L}\left(t_{h}^{\ast},t_{w}^{\ast},\lambda^{\ast}\right)}{\partial t_{h}}\left(t_{h}-t_{h}^{\ast}\right)+\frac{\partial\mathcal{L}\left(t_{h}^{\ast},t_{w}^{\ast},\lambda^{\ast}\right)}{\partial t_{w}}\left(t_{w}-t_{w}^{\ast}\right)\\
= & U\left(t_{h}^{\ast},t_{w}^{\ast};T\right).
\end{align*}
Hence, $\left(t_{h}^{\ast},t_{w}^{\ast},\lambda^{\ast}\right)$ exists
and is unique.
\end{proof}
Now consider the commuter's choice of departure time given the optimal
allocation of time. If the time constraint is binding, the mobile
activity will be performed only while travelling since all the remaining
time will be allocated to activities at either end of the trip. Hence,
at optimum the commuter departs at time $t_{h}^{\ast}$. On the other
hand, if the time constraint is non-binding, i.e., $\lambda^{\ast}=0$,
then the commuter undertakes the mobile activity while travelling
and at home and/or at work. In this case, the departure time may be
any time between $t_{h}^{\ast}$ and $Q-t_{w}^{\ast}-T$.%
\begin{comment}
Since travel time is deterministic, there is no need for the commuter
to give head start. As such the departure time can set in such a way
that it is aligned with the optimal time allocation. The optimal departure
time depends on whether or not the time constraint is binding. If
the time constraint is binding, time at the origin and destination
will be fully devoted to the home-based activity and the work-based
activity, respectively, with the mobile activity being carried out
only while travelling.
\end{comment}
{} This is so since the effective units of the mobile activity per unit
time is the same at home and at work.

The utility to an optimising commuter is
\begin{equation}
U^{\ast}\equiv U\left(t_{h}^{\ast},t_{w}^{\ast};T\right)=\max_{t_{h},t_{w},t_{h}+t_{w}+T\leq Q}U_{h}\left(t_{h}\right)+U_{m}\left(Q-\left(1-\alpha\right)T-t_{h}-t_{w}\right)+U_{w}\left(t_{w}\right),\label{eq:UStarDet}
\end{equation}
which is defined for a given trip duration and level of productivity
of in-vehicle time, and hence a change in these quantities can affect
the optimal utility. The change in optimal utility per unit change
in travel time is called the value of travel time. Similarly, the
effect of a change in the level of productivity of in-vehicle time
is an important consideration in efforts aimed at enhancing the usefulness
of travel time.

Since utility is optimised with respect to $\left(t_{h},t_{w}\right)$,
a change in the level of productivity of in-vehicle time does not
have a second-order effect on $U^{\ast}$. The effect on optimal utility
occurs fully through its impact on the effective units of the mobile
activity undertaken. The resulting effect positive implying that enhancing
the productivity of in-vehicle time is welfare improving. The gain
depends on the marginal utility from a unit of in-vehicle time and
the total trip duration. The longer the trip duration, the higher
is the benefit from enhanced productivity as the effect spreads across
longer duration.

\subsection{The value of travel time}

A change in trip duration affects optimal utility through its impact
on the time constraint and the effective units of the mobile activity
produced. Applying the envelope theorem on $U^{\ast}$, we obtain
that $\frac{\mathrm{d}U^{\ast}}{\mathrm{d}T}=\left(\alpha-1\right)U_{m}^{\prime}\left(Q-\left(1-\alpha\right)T-t_{h}^{\ast}-t_{w}^{\ast}\right)-\lambda^{\ast}\leq0$
where the inequality follows since $U_{m}^{\prime}\left(\cdot\right)>0$,
$\lambda^{\ast}\geq0$ and $\alpha\leq1$. This implies that a unit
reduction in travel time is worth $\frac{\mathrm{d}U^{\ast}}{\mathrm{d}T}$
to the commuter. Hence, the commuter will be indifferent between forgoing
$\frac{\mathrm{d}U^{\ast}}{\mathrm{d}T}$ units of money for a unit
reduction in travel time. This rate of trade-off between travel time
and money is called the value of travel time, which is given by:

\begin{equation}
-\frac{\mathrm{d}U^{\ast}}{\mathrm{d}T}=\left(1-\alpha\right)U_{m}^{\prime}\left(Q-\left(1-\alpha\right)T-t_{h}^{\ast}-t_{w}^{\ast}\right)+\lambda^{\ast}=U_{h}^{\prime}\left(t_{h}^{*}\right)-\alpha U_{m}^{\prime}\left(Q-\left(1-\alpha\right)T-t_{h}^{\ast}-t_{w}^{\ast}\right)\label{eq:VOT_det}
\end{equation}
Accordingly, the value of travel time is the sum of the resource value
of a unit of time, $\lambda^{\ast}$, and the gain from using this
saved time to carry out the mobile activity more productively at home
or work. Alternatively, it can be interpreted as the gain to the commuter
had the saved time been used to carry out the home-based or work-based
activity as opposed to undertaking the mobile activity while travelling.
\begin{defn}
(\textbf{VOT}) The \textbf{\textit{value of travel time}} refers to
maximum amount a commuter is willing-to-pay for a unit reduction in
travel time. Mathematically, it is equal to $-\frac{\mathrm{d}U^{\ast}}{\mathrm{d}T}$.
\end{defn}
The fact that $U_{h}^{\prime}\left(t_{h}^{*}\right)\geq U_{m}^{\prime}\left(t_{m}^{\ast}\right)$
implies that the commuter is willing to pay some amount for a unit
reduction in the mean travel time. The size this amount depends on
the status of the time constraint and the level of productivity of
in-vehicle time. If the time constraint is non-binding and in-vehicle
time is fully productive, then the value travel time will be zero.
On the other hand, if the constraint is binding, then the value of
travel time is positive even if travel time is fully productive. This
is so since, in this case, more time is allocated to the mobile activity
than desired. 
\begin{prop}
The value of travel time declines with the productivity of in-vehicle
time if the coefficient of risk aversion of $U_{m}$ is lower than
1, i.e., $\vartheta=-\frac{t_{m}^{\ast}U_{m}^{\prime\prime}\left(t_{m}^{\ast}\right)}{U_{m}^{\prime}\left(t_{m}^{\ast}\right)}<1$
where $t_{m}^{\ast}=t_{m}\left(t_{h}^{\ast},t_{w}^{\ast}\right)$.
\begin{proof}
We want to show that
\begin{align*}
\frac{\partial\left(-\frac{\mathrm{d}U^{\ast}}{\mathrm{d}T}\right)}{\partial\alpha} & =\left(1-\alpha\right)\left(T-\frac{\partial t_{h}^{\ast}}{\partial\alpha}-\frac{\partial t_{w}^{\ast}}{\partial\alpha}\right)U_{m}^{\prime\prime}\left(t_{m}^{\ast}\right)-U_{m}^{\prime}\left(t_{m}^{\ast}\right)+\frac{\partial\lambda^{\ast}}{\partial\alpha}<0
\end{align*}
provided that $\vartheta\equiv-\frac{t_{m}^{\ast}U_{m}^{\prime\prime}\left(t_{m}^{\ast}\right)}{U_{m}^{\prime}\left(t_{m}^{\ast}\right)}<1$.
To see this, differentiate the first-order conditions in (\ref{eq:foc_deterministic}):
\begin{align*}
U_{h}^{\prime\prime}\text{\ensuremath{\left(t_{h}\right)}}\frac{\partial t_{h}^{\ast}}{\partial\alpha}-U_{m}^{\prime\prime}\text{\ensuremath{\left(t_{m}^{\ast}\right)}}\left(T-\frac{\partial t_{h}^{\ast}}{\partial\alpha}-\frac{\partial t_{w}^{\ast}}{\partial\alpha}\right)-\frac{\partial\lambda^{\ast}}{\partial\alpha}= & 0\\
U_{w}^{\prime\prime}\text{\ensuremath{\left(t_{w}\right)}}\frac{\partial t_{w}^{\ast}}{\partial\alpha}-U_{m}^{\prime\prime}\text{\ensuremath{\left(t_{m}^{\ast}\right)}}\left(T-\frac{\partial t_{h}^{\ast}}{\partial\alpha}-\frac{\partial t_{w}^{\ast}}{\partial\alpha}\right)-\frac{\partial\lambda^{\ast}}{\partial\alpha}= & 0
\end{align*}
and solve for $\frac{\partial t_{w}^{\ast}}{\partial\alpha}$ to obtain
$\frac{\partial t_{w}^{\ast}}{\partial\alpha}=\frac{U_{h}^{\prime\prime}\text{\ensuremath{\left(t_{h}^{\ast}\right)}}}{U_{w}^{\prime\prime}\text{\ensuremath{\left(t_{w}^{\ast}\right)}}}\frac{\partial t_{h}^{\ast}}{\text{\ensuremath{\partial\alpha}}}$.
Substituting this in the above and manipulating, we have: 
\[
\frac{\partial t_{h}^{\ast}}{\partial\alpha}=\frac{\frac{\partial\lambda^{\ast}}{\partial\alpha}+TU_{m}^{\prime\prime}\left(t_{m}^{\ast}\right)}{U_{h}^{\prime\prime}\text{\ensuremath{\left(t_{h}^{\ast}\right)}}+U_{m}^{\prime\prime}\left(t_{m}^{\ast}\right)\left(1+\frac{U_{h}^{\prime\prime}\text{\ensuremath{\left(t_{h}^{\ast}\right)}}}{U_{w}^{\prime\prime}\text{\ensuremath{\left(t_{w}^{\ast}\right)}}}\right)}.
\]
Now, examine $\frac{\partial\left(-\frac{\mathrm{d}U^{\ast}}{\mathrm{d}T}\right)}{\partial\alpha}$
when the time constraint is binding and when it is not:
\begin{casenv}
\item Non-binding time constraint. \\
In this case, $\lambda^{\ast}=\frac{\partial\lambda^{\ast}}{\partial\alpha}=0$,
hence $\frac{\partial t_{h}^{\ast}}{\partial\alpha}>0$ and 
\begin{align*}
\frac{\partial\left(-\frac{\mathrm{d}U^{\ast}}{\mathrm{d}T}\right)}{\partial\alpha} & =\left(1-\alpha\right)T\frac{U_{h}^{\prime\prime}\text{\ensuremath{\left(t_{h}^{\ast}\right)}}U_{m}^{\prime\prime}\left(t_{m}^{\ast}\right)}{U_{h}^{\prime\prime}\text{\ensuremath{\left(t_{h}^{\ast}\right)}}+U_{m}^{\prime\prime}\left(t_{m}^{\ast}\right)\left(1+\frac{U_{h}^{\prime\prime}\text{\ensuremath{\left(t_{h}^{\ast}\right)}}}{U_{w}^{\prime\prime}\left(t_{w}^{\ast}\right)}\right)}-U_{m}^{\prime}\left(t_{m}^{\ast}\right)<0,
\end{align*}
Hence, the value of travel time decreases with the productivity of
in-vehicle time irrespective of the value of $\vartheta$.
\item Binding time constraint. \\
With binding constraint, we have $\frac{\partial t_{h}^{\ast}}{\partial\alpha}=\frac{\partial t_{w}^{\ast}}{\partial\alpha}=0$,
$t_{m}^{\ast}=\alpha T$ and $\frac{\partial\lambda}{\partial\alpha}=-TU_{m}^{\prime\prime}\left(t_{m}^{\ast}\right)$.
Thus, 
\[
\frac{\partial\left(-\frac{\mathrm{d}U^{\ast}}{\mathrm{d}T}\right)}{\partial\alpha}=-U_{m}^{\prime}\left(t_{m}^{\ast}\right)-t_{m}^{\ast}U_{m}^{\prime\prime}\left(t_{m}^{\ast}\right)<0,\mbox{ if }\vartheta=-\frac{t_{m}^{\ast}U_{m}^{\prime\prime}\left(t_{m}^{\ast}\right)}{U_{m}^{\prime}\left(t_{m}^{\ast}\right)}<1.
\]
\end{casenv}
Therefore, in general, the value of travel time decreases with the
productivity of in-vehicle time provided $\vartheta<1$.
\end{proof}
\end{prop}

\subsection{Example}

Consider a commuter with a time budget of $Q=1$ time units and trip
utility:
\begin{align*}
U\left(t_{h},t_{w};T\right) & =\ln\left(t_{h}\right)+\beta_{m}\ln\left(1-\left(1-\alpha\right)T-t_{h}-t_{w}\right)+\ln\left(t_{w}\right)
\end{align*}
where $\beta_{m}>0$. The commuter's problem is to 
\begin{gather*}
\max U\left(t_{h},t_{w};T\right)\\
\mbox{s.t. }t_{h}+t_{w}+T\leq1
\end{gather*}
The Lagrangian of the utility maximisation problem will be
\begin{gather*}
U\left(t_{h},t_{w};T\right)+\lambda\left(1-T-t_{h}-t_{w}\right)
\end{gather*}
with first-order conditions:
\begin{align*}
\frac{1}{t_{h}}-\frac{\beta_{m}}{1-\left(1-\alpha\right)T-t_{h}-t_{w}}-\lambda & =0\\
\frac{1}{t_{w}}-\frac{\beta_{m}}{1-\left(1-\alpha\right)T-t_{h}-t_{w}}-\lambda & =0\\
\lambda\left(1-T-t_{w}-t_{h}\right) & =0\\
\lambda,t_{h},t_{w} & \geq0.
\end{align*}

Solution:
\begin{casenv}
\item If the constraint is active, $t_{h}^{\ast}=t_{w}^{\ast}=\frac{1-T}{2}$,
$t_{m}^{\ast}=\alpha T$ and $\lambda^{\ast}=\frac{\left(2\alpha+\beta_{m}\right)T-\beta_{m}}{\alpha T\left(1-T\right)}$.
$\lambda^{\ast}>0$ holds if $T>\frac{\beta_{m}}{2\alpha+\beta_{m}}$.
\\
The constraint at this solution is
\[
2\frac{1-T}{2}+T=1
\]
which is True
\item If the constraint is inactive at the optimum, then $t_{h}^{\ast}=t_{w}^{\ast}=\frac{1-\left(1-\alpha\right)T}{2+\beta_{m}}$,
and hence $t_{m}^{\ast}=\beta_{m}\frac{1-\left(1-\alpha\right)T}{2+\beta_{m}}$.
That is, the total available time will be allocated across activities
based on the marginal productivity of time. The resulting optimal
utility is
\[
U^{\ast}=2\ln\left(\frac{1-\left(1-\alpha\right)T}{2+\beta_{m}}\right)+\beta_{m}\ln\left(\frac{\beta_{m}\left(1-\left(1-\alpha\right)T\right)}{2+\beta_{m}}\right)
\]
The constraint at this solution is
\begin{align*}
2\frac{1-\left(1-\alpha\right)T}{2+\beta_{m}}+T & <1\\
-2\left(1-\alpha\right)T+\left(2+\beta_{m}\right)T & <\beta_{m}\\
T & <\frac{\beta_{m}}{\beta_{m}+2\alpha}
\end{align*}
The value of travel time is
\[
-\frac{2\text{\ensuremath{\left(\alpha-1\right)}}+\beta_{m}\left(\alpha-1\right)}{1-\left(1-\alpha\right)T}>0.
\]
\end{casenv}
Therefore, the optimum depends on the travel time $T$.

\begin{figure}
\begin{centering}
\includegraphics{uStarAlpha}
\par\end{centering}
\centering{}\caption{Optimal utility as a function of $\alpha$}
\end{figure}


\section{Allocation of time and the values of travel time and reliability
with random travel time }

\subsection{Optimal time allocation with random trip duration}

Suppose travel time $T$ is a random variable with a distribution
bounded between $0$ and $Q$ \textit{such that the time constraint
is never exceeded}. We parametrise travel time in a convenient form
$T=\mu+\sigma X$, where $\mu$ is its mean, $\sigma$ its standard
deviation and $X$ is a standardised random variable with zero mean
and unit variance and a probability density function $f\left(X\right)$.
The bound on the distribution of $T$ implies that the distribution
of $X$ is also bounded between $-\frac{\mu}{\sigma}$ and $\frac{Q-\mu}{\sigma}$. 

Utility is stochastic since it depends on the unknown travel time,
$T$. Thus, the exact outcome for a given allocation of time $\left(t_{h},t_{w}\right)$
cannot be determined before the trip. We assume the commuter allocates
his/her total time across activities in view of a known travel time
distribution. The decision is taken sequentially. Firstly, time at
home is optimally allocated between the home-based and mobile activities
for any given travel duration and time devoted to the work-based activity.
Once travel time is realised upon arrival at work, the commuter optimally
allocates the remaining time between the work-based and mobile activities.

We solve the utility maximisation problem by backward induction: Conditional
on the information at the time of arrival at work, the commuter determines
the optimal time to be allocated to the work-based activity given
the realised travel time and the time devoted to the home-based and
mobile activities. Then, the optimal time for the home-based activity
is determined considering the distribution of travel times.

Upon arrival at the destination, the commuter has $Q-T-t_{d}$ time
units to be allocated between the mobile and work-based activities.
As travel time is realised, the commuter faces no uncertainty and
his/her problem amounts to:
\begin{gather*}
\max_{t_{w}}U_{m}\left(Q-\left(1-\alpha\right)\left(\mu+\sigma X\right)-t_{h}-t_{w}\right)+U_{w}\left(t_{w}\right)\\
\mbox{s.t. }t_{w}\leq Q-\mu-\sigma X-t_{d}.
\end{gather*}
The constraint indicates that the time that will be allocated to the
work-based activity cannot exceed the total available time at the
destination. The Lagrangian of the maximisation problem will be
\[
U_{m}\left(Q-\left(1-\alpha\right)\left(\mu+\sigma X\right)-t_{h}-t_{w}\right)+U_{w}\left(t_{w}\right)+\eta\left(Q-\mu-\sigma X-t_{d}-t_{w}\right),
\]
with first-order conditions for an optimum being

\begin{subequations}

\label{eq:tw_stage}
\begin{align}
U_{w}^{\prime}\left(t_{w}\right)-U_{m}^{\prime}\left(Q-\left(1-\alpha\right)\left(\mu+\sigma X\right)-t_{h}-t_{w}\right)-\eta & =0\label{eq:stage2_wrt_tw}\\
\eta\left(Q-\mu-\sigma X-t_{d}-t_{w}\right) & =0\label{eq:stage2_compl}\\
\eta,t_{w} & \geq0\label{eq:stage2_nonnegative}
\end{align}

\end{subequations}

If the constraint is non-binding, then the optimal time allocated
to the work-based activity, $\hat{t}_{w}$, will be that which equates
the marginal utility of time in the work-based and mobile activities:
\[
\hat{t}_{w}=\left\{ t_{w}:U_{w}^{\prime}\left(t_{w}\right)-U_{m}^{\prime}\left(Q-\left(1-\alpha\right)\left(\mu+\sigma X\right)-t_{h}-t_{w}\right)=0\right\} .
\]
If however the constraint is binding, then all the time at the destination
will be used to carry out the work-based activity, hence $\hat{t}_{w}=Q-t_{d}-\mu-\sigma X$.
In this case, the marginal utility of time in the work-based activity
will be higher than that in the mobile activity. This is so since,
if this was not the case, then the commuter will be better off allocating
some of the time at the destination to the mobile activity. If there
exists a pair $\left(\hat{t}_{w},\hat{\eta}\right)$ satisfying the
first-order conditions in (\ref{eq:tw_stage}), then the optimal utility
will be:

\[
V\left(\hat{t}_{w};t_{h}\right)\equiv U_{m}\left(Q-\left(1-\alpha\right)\left(\mu+\sigma X\right)-t_{h}-\hat{t}_{w}\right)+U_{w}\left(\hat{t}_{w}\right)+\hat{\eta}\left(Q-\mu-\sigma X-t_{d}-\hat{t}_{w}\right).
\]

Given the optimal allocation of time at work, the commuter will determine
the optimal allocation of his/her total time at the origin between
the mobile and home-based activities. This decision is made in light
of the travel time distribution as the exact travel time is not known
before the trip is taken. The commuter chooses the time for the home-based
activity in order to maximise expected utility:
\begin{gather*}
\max_{t_{h}}E\left[U_{h}\left(t_{h}\right)+V\left(\hat{t}_{w};t_{h}\right)\right]\\
\mbox{s.t. }t_{h}\leq t_{d}
\end{gather*}
where the expectation is over all possible values of the standardised
travel time, $X$. The Lagrangian of the utility maximisation problem
can be given as 
\[
E\left[U_{h}\left(t_{h}\right)+U_{m}\left(Q-\left(1-\alpha\right)\left(\mu+\sigma X\right)-t_{h}-\hat{t}_{w}\right)+U_{w}\left(\hat{t}_{w}\right)+\phi\left(t_{h}-t_{d}\right)\right]
\]
with first-order conditions

\begin{subequations}\label{eq:th_stage}
\begin{align}
E\left[U_{h}^{\prime}\left(t_{h}\right)-U_{m}^{\prime}\left(Q-\left(1-\alpha\right)\left(\mu+\sigma X\right)-t_{h}-\hat{t}_{w}\right)-\phi\right] & =0\label{eq:stage1_wrt_th}\\
\phi,t_{h} & \geq0\label{eq:stage1_lambda}\\
\phi\left(t_{d}-t_{h}\right) & =0\label{eq:stage1_lambdai_const}
\end{align}

\end{subequations}

An optimal allocation exists if there is a quartet $\left(\hat{t}_{h},\hat{t}_{w},\hat{\eta},\hat{\phi}\right)$
satisfying the conditions in (\ref{eq:tw_stage}) and (\ref{eq:th_stage}).
The existence and uniqueness of such an optimum is given in the following
theorem:
\begin{thm}
\label{thm:existence_stochastic}Suppose $T<Q$ and that, for $i=\left\{ h,w,m\right\} $,
each utility component $U_{i}$ is increasing, strictly concave and
twice-continuously differentiable. Then, there exists an optimal time
allocation, $\left(\hat{t}_{h},\hat{t}_{w}\right)$, and multipliers
$\hat{\eta}$ and $\hat{\phi}$ satisfying (\ref{eq:tw_stage}) and
(\ref{eq:th_stage}). This optimum is is unique.
\end{thm}
\begin{proof}
\textcolor{brown}{The proof for existence and uniqueness at each stage
straightforwardly follows the proof of Theorem 1. The main claim to
be proved here is the existence and uniqueness of an optimum jointly
in the two stages}. We need to show that $V\left(\hat{t}_{w};t_{h}\right)$
is concave in $t_{h}$. Since $U_{h}$ and $V$ are strictly concave
in $t_{h}$, the objective function in the choice for an optimal $t_{h}$
is also concave. Thus, it has an optimal allocation, say $\left(\hat{t}_{h},\hat{t}_{w}\right)$.
Now, The following is needed to prove the existence of an optimal
allocation $\left(\hat{t}_{h},\hat{t}_{w}\right)$ and multipliers
$\left(\hat{\eta},\hat{\phi}\right)$: (a) $V$ should be concave
in $t_{h}$, which is the case for a given $t_{d}$; (b)
\end{proof}
The optimal allocation and departure times can differ depending on
the status of the two time constraints. If the commuter has binding
constraints at both ends of the trip, then $\hat{t}_{w}=Q-t_{d}-\mu-\sigma X$
and the mobile activity will be carried out only while travelling,
hence $\hat{t}_{m}=\alpha\left(\mu+\sigma X\right)$. In this case,
the expected marginal utility of time in the mobile activity will
be lower than that in the home-based or work-based activities. In
addition, $\hat{t}_{h}=t_{d}$, hence the commuter will depart as
soon as he/she spent the last minute of the time allocated to the
home-based activity. This is also the case with binding constraint
at home irrespective of the status of the constraint at work.

On the other hand, if both the constraints are non-binding, then the
marginal expected utility of time will be equalised across the three
activities. This is analogous to the case with deterministic travel
time. In this case, the optimal departure time can be any moment between
$\hat{t}_{h}$ and $Q-\hat{t}_{w}-\mu-\sigma X$.\footnote{Since the exact travel time is unknown before the trip, the optimal
departure time may be determined based on a predicted travel time
such as the mean, median or other features of the travel time distribution.} This is also the case if the constraint at home is non-binding irrespective
of whether the constraint at work is binding. 

If the commuter has binding constraint at work but non-binding constraint
at home, i.e., $\eta>0$ and $\phi=0$, then optimality requires that
the expected marginal utility of time in the work-based activity be
higher than that in the mobile or home-based activities. In this case,
the optimal departure time can be any moment between $\hat{t}_{h}$
and $Q-\hat{t}_{w}-\mu-\sigma X$. Conversely, in the case where the
constraint at home is binding while that at work is not, the optimal
departure time will be $\hat{t}_{h}$ units of time in the home-based
activity, and the expected marginal utility of time in the home-based
activity will be higher than that in the mobile or work-based activities.

The optimal expected utility: 
\[
W\equiv E\left[U\left(\hat{t}_{h},\hat{t}_{w}\right)\right]=E\left[U_{h}\left(\hat{t}_{h}\right)+U_{m}\left(Q-\left(1-\alpha\right)\left(\mu+\sigma X\right)-\hat{t}_{h}-\hat{t}_{w}\right)+U_{w}\left(\hat{t}_{w}\right)\right],
\]
increases with the productivity of in-vehicle time, $\alpha$. Hence,
as was the case with deterministic travel time, an improvement in
the productivity of in-vehicle time is welfare improving.

\subsection{Example}

\subsection{The value of time and reliability with random travel time}

\subsubsection*{The value of travel time}

The value of travel time can be obtained by enveloping the optimal
expected utility:

\begin{alignat*}{1}
-\frac{\mathrm{d}W}{\mathrm{d}\mu} & =E\left[U_{w}^{\prime}\left(\hat{t}_{w}\right)-\alpha U_{m}^{\prime}\left(Q-\left(1-\alpha\right)\left(\mu+\sigma X\right)-\hat{t}_{h}-\hat{t}_{w}\right)\right]
\end{alignat*}
The value of travel time is the difference in the expected marginal
utility of time in the work-based activity and the mobile activity,
with the latter being weighted by the productivity of in-vehicle time
relative to time at work or at home. 

If the constraint at the destination is non-binding and travel time
is fully productive, then the value of travel time will be zero. However,
if the constraint is binding, then the value travel time will be positive
even if travel time is fully productive. This result is similar to
the case with deterministic travel time. Note that the value of time
is not influenced by the status of the time constraint at the origin.
\textcolor{brown}{Why?}
\begin{prop}
The value of travel time decreases with the productivity of in-vehicle
time provided that $\vartheta<1$. 
\end{prop}
\begin{proof}
We want to show that 

\begin{align*}
\frac{\partial\left(-\frac{\mathrm{d}W}{\mathrm{d}\mu}\right)}{\partial\alpha}= & E\left[\frac{\partial\hat{t}_{w}}{\partial\alpha}U_{w}^{\prime\prime}\left(\hat{t}_{w}\right)+\alpha\left(\frac{\partial\hat{t}_{h}}{\partial\alpha}+\frac{\partial\hat{t}_{w}}{\partial\alpha}-\mu-\sigma X\right)U_{m}^{\prime\prime}\left(\hat{t}_{m}\right)-U_{m}^{\prime}\left(\hat{t}_{m}\right)\right]<0
\end{align*}
if $\vartheta<1$. The value of reliability declines with the productivity
of in-vehicle time if $\frac{\partial\hat{t}_{h}}{\partial\alpha}\geq0$,
$\frac{\partial\hat{t}_{w}}{\partial\alpha}\geq0$ and $-\alpha\left(\mu+\sigma X\right)U_{m}^{\prime\prime}\left(\hat{t}_{m}\right)-U_{m}^{\prime}\left(\hat{t}_{m}\right)<0$.
To state the latter in terms of $\vartheta$, define $\xi\equiv-\alpha\left(\mu+\sigma X\right)U_{m}^{\prime\prime}\left(\hat{t}_{m}\right)-U_{m}^{\prime}\left(\hat{t}_{m}\right)$
and $\hat{t}_{m}=\alpha\left(\mu+\sigma X\right)+\epsilon$ where
$\epsilon\geq0$. Thus,
\begin{align*}
\xi & =-\left(\alpha\left(\mu+\sigma X\right)+\epsilon\right)U_{m}^{\prime\prime}\left(\alpha\tau+\epsilon\right)-U_{m}^{\prime}\left(\alpha\left(\mu+\sigma X\right)+\epsilon\right)+\epsilon U_{m}^{\prime\prime}\left(\alpha\left(\mu+\sigma X\right)+\epsilon\right)<0,
\end{align*}
if $\vartheta+\frac{\epsilon U_{m}^{\prime\prime}\left(\alpha\left(\mu+\sigma X\right)+\epsilon\right)}{U_{m}^{\prime}\left(\alpha\left(\mu+\sigma X\right)+\epsilon\right)}<1$,
which holds provided that $\vartheta<1$. Thus, it remains to show
that $\frac{\partial\hat{t}_{h}}{\partial\alpha}\geq0$ and $\frac{\partial\hat{t}_{w}}{\partial\alpha}\geq0$.
To see this, differentiate the first-order conditions in (\ref{eq:tw_stage})
and (\ref{eq:th_stage}) to find that:

\begin{align*}
\frac{\partial\hat{t}_{h}}{\partial\alpha}= & \frac{\frac{\partial\hat{t}_{w}}{\partial\alpha}U_{w}^{\prime\prime}\left(\hat{t}_{w}\right)+\frac{\partial\hat{\phi}}{\partial\alpha}-\frac{\partial\hat{\eta}}{\partial\alpha}}{U_{h}^{\prime\prime}\left(\hat{t}_{h}\right)}\\
\frac{\partial\hat{t}_{w}}{\partial\alpha}= & \frac{\frac{\partial\hat{\eta}}{\partial\alpha}+\left(\mu+\sigma X-\frac{\frac{\partial\hat{\phi}}{\partial\alpha}-\frac{\partial\hat{\eta}}{\partial\alpha}}{U_{h}^{\prime\prime}\left(\hat{t}_{h}\right)}\right)U_{m}^{\prime\prime}\left(\hat{t}_{m}\right)}{U_{w}^{\prime\prime}\left(\hat{t}_{w}\right)+U_{m}^{\prime\prime}\left(\hat{t}_{m}\right)\left(1+\frac{U_{w}^{\prime\prime}\left(\hat{t}_{w}\right)}{U_{h}^{\prime\prime}\left(\hat{t}_{h}\right)}\right)}
\end{align*}
Now, examine what happens to $\frac{\partial\hat{t}_{h}}{\partial\alpha}$
and $\frac{\partial\hat{t}_{w}}{\partial\alpha}$ under the following
cases: 
\begin{casenv}
\item If both constraints are binding, then $\frac{\partial\hat{t}_{w}}{\partial\alpha}=\frac{\partial\hat{t}_{h}}{\partial\alpha}=0$. 
\item If both constraints are non-binding, then $\frac{\partial\hat{\phi}}{\partial\alpha}=\frac{\partial\hat{\eta}}{\partial\alpha}=0$,
hence $\frac{\partial\hat{t}_{h}}{\partial\alpha}>0$ and $\frac{\partial\hat{t}_{w}}{\partial\alpha}>0$.
\item If $\hat{\phi}>0$ and $\hat{\eta}=0$, then $\frac{\partial\hat{t}_{h}}{\partial\alpha}=\frac{\partial\hat{\eta}}{\partial\alpha}=0$
and $\frac{\partial\hat{t}_{w}}{\partial\alpha}\geq0$. 
\item If $\hat{\phi}=0$ and $\hat{\eta}>0$, then $\frac{\partial\hat{t}_{w}}{\partial\alpha}=\frac{\partial\hat{\phi}}{\partial\alpha}=0$
and $\frac{\partial\hat{t}_{h}}{\partial\alpha}\geq0$.
\end{casenv}
Therefore, since $\frac{\partial\hat{t}_{h}}{\partial\alpha}\geq0$
and $\frac{\partial\hat{t}_{w}}{\partial\alpha}\geq0$, $\frac{\partial\left(-\frac{\mathrm{d}W}{\mathrm{d}\mu}\right)}{\partial\alpha}<0$
if $\vartheta<1$.
\end{proof}

\subsubsection*{The value of reliability }
\begin{defn}
(\textbf{VOR}) The \textbf{\textit{value of reliability}} refers to
the maximum amount a commuter is willing-to-pay for a unit reduction
in standard deviation of travel time: the quantity $-\frac{\mathrm{d}W}{\mathrm{d}\sigma}$.
\end{defn}
The value of reliability can be derived by differentiating the optimal
expected utility with respect to the standard deviation of travel
time $\sigma$: 
\begin{align*}
-\frac{\mathrm{d}W}{\mathrm{d}\sigma} & =E\left[\left(1-\alpha\right)XU_{m}^{\prime}\left(Q-\left(1-\alpha\right)\left(\mu+\sigma X\right)-\hat{t}_{h}-\hat{t}_{w}\right)\right].
\end{align*}
The resulting value indicates the weighted gain in expected utility
as the commuter performs the mobile activity at either end of the
trip as opposed to while travelling, where the weight is the standardised
travel time. If travel time is as productive as time at home or at
work, then there is no gain in undertaking the mobile activity at
home or at work. As a result, the value of reliability is zero if
travel time is fully productive. Irrespective of the status of the
two time constraints, the value of reliability is non-negative irrespective
of the status of the time constraints. \textcolor{brown}{One would
expect the value of reliability to be higher with a binding constraint
at the destination?} 
\begin{prop}
The value of reliability is non-negative and it declines with the
productivity of in-vehicle time if $\frac{\partial\hat{t}_{m}}{\partial\alpha}\geq0$. 
\end{prop}
\begin{proof}
To show that the value of reliability is non-negative, let

\[
U_{m}^{\prime}\left(Q-\left(1-\alpha\right)\left(\mu+\sigma X\right)-\hat{t}_{h}-\hat{t}_{w}\right)=\begin{cases}
v_{1} & \mbox{ if X >0}\\
v_{2} & \mbox{ if X\ensuremath{\leq0}}
\end{cases}
\]
where $v_{1}$ and $v_{2}$ are positive constants with $v_{1}\geq v_{2}$
since $\frac{\partial U_{m}^{\prime}\left(Q-\left(1-\alpha\right)\left(\mu+\sigma X\right)-\hat{t}_{h}-\hat{t}_{w}\right)}{\partial X}=-\left(1-\alpha\right)\sigma U_{m}^{\prime\prime}\left(\cdot\right)\geq0$.
Using this, the value of reliability can be given as,
\begin{align*}
-\frac{\mathrm{d}W}{\mathrm{d}\sigma} & =E\left[\left(1-\alpha\right)XU_{m}^{\prime}\left(Q-\left(1-\alpha\right)\left(\mu+\sigma X\right)-\hat{t}_{h}-\hat{t}_{w}\right)\vert X>0\right]\\
 & \qquad+E\left[\left(1-\alpha\right)XU_{m}^{\prime}\left(Q-\left(1-\alpha\right)\left(\mu+\sigma X\right)-\hat{t}_{h}-\hat{t}_{w}\right)\vert X\leq0\right]\\
 & =E\left[\left(1-\alpha\right)v_{1}X\vert X>0\right]+E\left[\left(1-\alpha\right)v_{2}X\vert X\leq0\right]
\end{align*}
Noting that $E\left[X\right]=0$ such that $E\left[X\vert X\leq0\right]=-E\left[X\vert X>0\right]$,
we have
\begin{align*}
-\frac{\mathrm{d}W}{\mathrm{d}\sigma} & =\left(1-\alpha\right)v_{1}E\left[X\vert X>0\right]+\left(1-\alpha\right)v_{2}E\left[X\vert X\leq0\right]\\
 & =\left(1-\alpha\right)v_{1}E\left[X\vert X>0\right]-\left(1-\alpha\right)v_{2}E\left[X\vert X>0\right]\\
 & =\left(1-\alpha\right)\left(v_{1}-v_{2}\right)E\left[X\vert X>0\right]\\
 & \geq0
\end{align*}
since $v_{1}\geq v_{2}$ and $E\left[X\vert X>0\right]$. Therefore,
the value of reliability is non-negative.

Moreover, we intend to show that

\begin{align*}
\frac{\partial\left(-\frac{dW}{d\sigma}\right)}{\partial\alpha}\text{=} & E\left[X\left(\left(1-\alpha\right)\left(\mu+\sigma X-\frac{\partial\hat{t}_{h}}{\partial\alpha}-\frac{\partial\hat{t}_{w}}{\partial\alpha}\right)U_{m}^{\prime\prime}\left(\hat{t}_{m}\right)-U_{m}^{\prime}\left(\hat{t}_{m}\right)\right)\right]<0
\end{align*}
if $\frac{\partial\hat{t}_{m}}{\partial\alpha}\geq0$.\footnote{Alternatively, the requirement for the claim can be stated as 
\[
\left(\frac{\partial\hat{t}_{h}}{\partial\alpha}+\frac{\partial\hat{t}_{w}}{\partial\alpha}\right)\pi<1
\]
where $\pi=-\frac{U_{m}^{\prime\prime}\left(\hat{t}_{m}\right)}{U_{m}^{\prime}\left(\hat{t}_{m}\right)}$
is the coefficient of absolute risk aversion of $U_{m}$.} Since $\frac{\partial U_{m}^{\prime}\left(\hat{t}_{m}\right)}{\partial X}\geq0$,
the above inequality holds if $\mu+\sigma X-\frac{\partial\hat{t}_{h}}{\partial\alpha}-\frac{\partial\hat{t}_{w}}{\partial\alpha}\geq0$,
which is the case if $\frac{\partial\hat{t}_{m}}{\partial\alpha}=\mu+\sigma X-\frac{\partial\hat{t}_{h}}{\partial\alpha}-\frac{\partial\hat{t}_{w}}{\partial\alpha}\geq0$.
\end{proof}
An increase in the productivity of in-vehicle time reduces the $\left(1-\alpha\right)$
and hence also the value of reliability, but its effect through $t_{m}$
depends on if and how much the optimal allocation of time changes
due to a change in $\alpha$. With binding time constraints at both
ends of the trip, $\frac{\partial t_{m}}{\partial\alpha}=\mu+\sigma X>0$
since the time optimally allocated to the home-based and work-based
activities will remain unchanged. Thus, the value of reliability declines
with the productivity of in-vehicle time irrespective of the value
of $\frac{\partial t_{m}}{\partial\alpha}$. However, when one or
both constraints are inactive, a change in productivity of in-vehicle
time also affects the value of reliability through its effect on the
optimal allocation of time. Accordingly, an increase in $\alpha$
increases $t_{h}$ or $t_{w}$ or both. As a result, the value of
reliability declines with the productivity of in-vehicle time only
if $\frac{\partial\hat{t}_{h}}{\partial\alpha}+\frac{\partial\hat{t}_{w}}{\partial\alpha}\leq\mu+\sigma X$.

\clearpage{}

Showing the existence of an optimal time allocation follows directly
from the proof of Theorem 1 if the optimal expected utility function:

\[
V\left(\hat{t}_{w};t_{h}\right)\equiv U_{m}\left(Q-\left(1-\alpha\right)\left(\mu+\sigma X\right)-t_{h}-\hat{t}_{w}\right)+U_{w}\left(\hat{t}_{w}\right)+\hat{\eta}\left(Q-\mu-\sigma X-t_{d}-\hat{t}_{w}\right).
\]
is concave in $t_{h}$. Since utility is continuous in $t_{h}$, concavity
follows if $\frac{\partial}{\partial t_{h}}\left(\frac{\partial V}{\partial t_{h}}\right)$.
The first derivative is:
\begin{align*}
\frac{\partial V}{\partial t_{h}}= & U_{m}^{\prime}\left(Q-\left(1-\alpha\right)\left(\mu+\sigma X\right)-t_{h}-\hat{t}_{w}\right)\left(-1-\frac{\hat{t}_{w}}{\partial t_{h}}\right)+U_{w}^{\prime}\left(\hat{t}_{w}\right)\frac{\hat{t}_{w}}{\partial t_{h}}+\hat{\eta}\left(-\frac{t_{d}}{\partial t_{h}}-\frac{\hat{t}_{w}}{\partial t_{h}}\right)\\
= & -U_{m}^{\prime}\left(Q-\left(1-\alpha\right)\left(\mu+\sigma X\right)-t_{h}-\hat{t}_{w}\right)<0
\end{align*}
hence,
\begin{align*}
\frac{\partial}{\partial t_{h}}\left(\frac{\partial V}{\partial t_{h}}\right) & =\frac{\partial}{\partial t_{h}}\left(-U_{m}^{\prime}\left(Q-\left(1-\alpha\right)\left(\mu+\sigma X\right)-t_{h}-\hat{t}_{w}\right)\right)\\
 & =U_{m}^{\prime\prime}\left(Q-\left(1-\alpha\right)\left(\mu+\sigma X\right)-t_{h}-\hat{t}_{w}\right)\left(1+\frac{\partial\hat{t}_{w}}{\partial t_{h}}\right)
\end{align*}
Now, if $\hat{\eta}=0$ then from the first-order condition, we have
$U_{w}^{\prime}\left(\hat{t}_{w}\right)=U_{m}^{\prime}\left(Q-\left(1-\alpha\right)\left(\mu+\sigma X\right)-t_{h}-\hat{t}_{w}\right)$
and hence
\[
\frac{\partial\hat{t}_{w}}{\partial t_{h}}=-\frac{U_{m}^{\prime\prime}\left(Q-\left(1-\alpha\right)\left(\mu+\sigma X\right)-t_{h}-\hat{t}_{w}\right)}{U_{w}^{\prime\prime}\left(\hat{t}_{w}\right)+U_{m}^{\prime\prime}\left(Q-\left(1-\alpha\right)\left(\mu+\sigma X\right)-t_{h}-\hat{t}_{w}\right)}>0
\]
If $\hat{\eta}>0$ then $t_{d}+t_{w}+T-Q=0$ and hence $\frac{\partial t_{d}}{\partial t_{h}}+\frac{\partial t_{w}}{\partial t_{h}}=0$
if $\frac{\partial t_{d}}{\partial t_{h}}=0$ then $\frac{\partial t_{w}}{\partial t_{h}}=0$.
Hence, if $\frac{\partial t_{d}}{\partial t_{h}}$

The conclusion is that $V$is concave in $t_{h}$

\clearpage{}
\begin{defn}
Consider defining the following
\end{defn}
\begin{enumerate}
\item Time allocation: A pair $\left(t_{h},t_{w}\right)$ indicating a commuter's
choice of time allocated to the home-based and work-based activities.
\item A time allocation is said to be \textbf{\textit{feasible}} if the
time allocated to the home-based and work-based activities does not
exceed the commuter's time budget less travel time.
\item A continuously differentiable function $f$ is \textbf{\textit{concave}}
if $f^{\prime\prime}<0$.
\item A constraint set is said to be \textbf{\textit{convex}} if a convex
mixture of any two feasible points within the set is also feasible.
\item The \textbf{\textit{value of travel time}} is a commuter's maximum
willing-to-pay for a unit reduction in travel time.
\item The \textbf{\textit{value of reliability}} is the maximum willing-to-pay
for a unit reduction in standard deviation of travel time.
\end{enumerate}

\end{document}
